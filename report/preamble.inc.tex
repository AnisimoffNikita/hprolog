\sloppy

% Настройки стиля ГОСТ 7-32
% Для начала определяем, хотим мы или нет, чтобы рисунки и таблицы нумеровались в пределах раздела, или нам нужна сквозная нумерация.
\EqInChapter % формулы будут нумероваться в пределах раздела
\TableInChapter % таблицы будут нумероваться в пределах раздела
\PicInChapter % рисунки будут нумероваться в пределах раздела
\usepackage{slashbox}

% Добавляем гипертекстовое оглавление в PDF
\usepackage[
bookmarks=true, colorlinks=true, unicode=true,
urlcolor=black,linkcolor=black, anchorcolor=black,
citecolor=black, menucolor=black, filecolor=black,
]{hyperref}

% Изменение начертания шрифта --- после чего выглядит таймсоподобно.
% apt-get install scalable-cyrfonts-tex

\IfFileExists{cyrtimes.sty}
    {
        \usepackage{cyrtimespatched}
    }
    {
        % А если Times нету, то будет CM...
    }

\usepackage{graphicx}   % Пакет для включения рисунков

% С такими оно полями оно работает по-умолчанию:
% \RequirePackage[left=20mm,right=10mm,top=20mm,bottom=20mm,headsep=0pt]{geometry}

\geometry{right=20mm}
\geometry{left=30mm}


% Пакет Tikz
\usepackage{tikz}
\usetikzlibrary{arrows,positioning,shadows}
\usepackage{pgfplots}

% Произвольная нумерация списков.
\usepackage{enumerate}

% ячейки в несколько строчек
\usepackage{multirow}

% itemize внутри tabular
\usepackage{paralist,array}

% Центрирование подписей к плавающим окружениям
\usepackage[justification=centering]{caption}

% My

\usepackage[export]{adjustbox}
\usepackage{float}

\usepackage{subcaption}
\usepackage{comment}
\usepackage{tabularx}





%%
\usepackage{ifthen}
\usepackage{url}
\usepackage{listings}
\usepackage{rotating}
\usepackage{afterpage}

\usepackage{tocvsec2}


%% caption less table to specify restful primitives
\newcommand{\restful}[1]
{
	\begin{center}
		\begin{tabular}{l p{12cm}}
			\hline
			#1
			\hline
		\end{tabular}
	\end{center}
	\vspace{6pt}
}


%% caption less table to specify restful uri
\newcommand{\routes}[1]
{
	\begin{center}
		\begin{tabular}{p{\textwidth}}
			\hline
			#1
			\hline
		\end{tabular}
	\end{center}
	\vspace{6pt}
}
%
%
%

\newcommand{\mimetype}[2]{#1 & #2 \\\noalign{\smallskip}}
\newcommand{\head}[2]{#1 & #2 \\\noalign{\smallskip}}
\newcommand{\uri}[2]{\url{#1} \\ #2 \vspace{8pt} \\\noalign{\smallskip}}
\newcommand{\env}[2]{#1 & #2 \\\noalign{\smallskip}}
%

%%% resource definition table
\newcommand{\resource}[1]
{
	\begin{center}
		\begin{tabular}{l l p{12cm}}
			\hline
			#1
			\hline
		\end{tabular}
	\end{center}
	\vspace{6pt}
}
\newcommand{\attr}[3]{{\tt #1} & #2 & #3 \\\noalign{\smallskip}}

%
%
\newcommand{\request}[6]
{
	\begin{center}
		\begin{tabular}{l p{12cm}}
			\hline
			Метод:  & #1 \url{#2} \\\noalign{\smallskip}
			& #3     \vspace{4pt}\\ 
			Запрос:  & #4 \vspace{4pt}\\
			Ответ:  & #5 \vspace{4pt}\\
			Статус:   & #6 \\
			\hline
		\end{tabular}
	\end{center}
	\vspace{6pt}
}

\newcommand{\sep}{\\\noalign{\smallskip} &}

\newcommand{\status}[1]
{
	\ifthenelse{\equal{#1}{100}}{100 Continue}{}%
	\ifthenelse{\equal{#1}{101}}{101 Switching Protocols}{}%
	\ifthenelse{\equal{#1}{200}}{200 OK}{}%
	\ifthenelse{\equal{#1}{201}}{201 Created}{}%
	\ifthenelse{\equal{#1}{202}}{202 Accepted}{}%
	\ifthenelse{\equal{#1}{203}}{203 Non-Authoritative Information}{}%
	\ifthenelse{\equal{#1}{204}}{204 No Content}{}%
	\ifthenelse{\equal{#1}{205}}{205 Reset Content}{}%
	\ifthenelse{\equal{#1}{206}}{206 Partial Content}{}%
	\ifthenelse{\equal{#1}{300}}{300 Multiple Choices}{}%
	\ifthenelse{\equal{#1}{301}}{301 Moved Permanently}{}%
	\ifthenelse{\equal{#1}{302}}{302 Found}{}%
	\ifthenelse{\equal{#1}{303}}{303 See Other}{}%
	\ifthenelse{\equal{#1}{304}}{304 Not Modified}{}%
	\ifthenelse{\equal{#1}{307}}{307 Temporary Redirect}{}%
	\ifthenelse{\equal{#1}{302}}{302 Found}{}%
	\ifthenelse{\equal{#1}{400}}{400 Bad Request}{}%
	\ifthenelse{\equal{#1}{401}}{401 Unauthorized}{}%
	\ifthenelse{\equal{#1}{402}}{402 Payment Required}{}%
	\ifthenelse{\equal{#1}{403}}{403 Forbidden}{}%
	\ifthenelse{\equal{#1}{404}}{404 Not Found}{}%
	\ifthenelse{\equal{#1}{405}}{405 Method Not Allowed}{}%
	\ifthenelse{\equal{#1}{406}}{406 Not Acceptable}{}%
	\ifthenelse{\equal{#1}{407}}{407 Proxy Authentication Required}{}%
	\ifthenelse{\equal{#1}{408}}{408 Request Timeout}{}%
	\ifthenelse{\equal{#1}{409}}{409 Conflict}{}%
	\ifthenelse{\equal{#1}{410}}{410 Gone}{}%
	\ifthenelse{\equal{#1}{411}}{411 Length Required}{}%
	\ifthenelse{\equal{#1}{412}}{412 Precondition Failed}{}%
	\ifthenelse{\equal{#1}{413}}{413 Request Entity Too Large}{}%
	\ifthenelse{\equal{#1}{414}}{414 Request-URI Too Long}{}%
	\ifthenelse{\equal{#1}{415}}{415 Unsupported Media Type}{}%
	\ifthenelse{\equal{#1}{416}}{416 Requested Range Not Satisfiable}{}%
	\ifthenelse{\equal{#1}{417}}{417 Expectation Failed}{}%
	\ifthenelse{\equal{#1}{422}}{422 Unprocessable Entity}{}%
	\ifthenelse{\equal{#1}{500}}{500 Internal Server Error}{}%
	\ifthenelse{\equal{#1}{501}}{501 Not Implemented}{}%
	\ifthenelse{\equal{#1}{502}}{502 Bad Gateway}{}%
	\ifthenelse{\equal{#1}{503}}{503 Service Unavailable}{}%
	\ifthenelse{\equal{#1}{504}}{504 Gateway Timeout}{}%
	\ifthenelse{\equal{#1}{505}}{505 HTTP Version Not Supported}{}%
}

\newcommand{\httpcode}[2]{\status{#1} & #2 \\\noalign{\smallskip}}

\newcommand{\example}[1]{\noindent {\bf #1}}
