
\begin{thebibliography}{9}
	 \bibitem{sql} Введение в системы баз данных: учебное пособие / Бураков П.В., Петров В.Ю.  -- Санкт-Петербург, 2010 -- 128c.
	 
	 \bibitem{sql2} Назначение и основные понятия БД.  [Электронный ресурс] Режим доступа: http://life-prog.ru/1\_32727\_naznachenie-i-osnovnie-ponyatiya-bd.html , свободный.
	 
	 \bibitem{sqlnsql} Базы данных SQL, NoSQL и различия в моделях баз данных [Электронный ресурс]. Режим доступа: http://devacademy.ru/posts/sql-nosql/, свободный.
	 
	 
	 \bibitem{vuz} Информационные системы: Учебник для вузов. /Избачков Юрий Сергеевич, Петров Владимир Николаевич, Васильев Александр Алексеевич, Телина Ирина Сергеевна. — 3-е изд. — Питер, 2010. — 544с.
	 
	 \bibitem{sqlnsql3} SQL или NoSQL — вот в чём вопрос
	 [Электронный ресурс]. Режим доступа: https://habr.com/company/ruvds/blog/324936/, свободный.
	 
	 \bibitem{sqlnsql2} SQL и NoSQL: разбираемся в основных моделях баз данных [Электронный ресурс]. Режим доступа: https://tproger.ru/translations/sql-nosql-
	 database-models/, свободный.
	 
	 
	 \bibitem{yac} О сервисе Яндекс.Контест. [Электронный ресурс] Режим доступа: https://contest.yandex.ru/about/ , свободный.
	 \bibitem{certm} О сервисе Сертификация Mail.ru. [Электронный ресурс] Режим доступа: https://certification.mail.ru/about/ , свободный.
	 \bibitem{stepik} О сервисе Stepik URL.  [Электронный ресурс] Режим доступа: https://welcome.stepik.org/ru/about , свободный.
	 
	 \bibitem{mdb} Бэнкер Кайл MongoDB в действии /Бэнкер Кайл. — 1-е изд. — Москва: ДМК-пресс, 2017. — 394с.
	 
	 \bibitem{haskell} Изучаем Haskell /Мена А.С. -- Санкт-Петербург: Питер, 2015. -- 464 pp.
	 
	 \bibitem{s} servant: A family of combinators for defining webservices APIs [Электронный ресурс]. Режим доступа: https://hackage.haskell.org/package/servant, свободный.
	 
	 \bibitem{p} Persistent и работа с базами данных в Yesod
	 [Электронный ресурс]. Режим доступа: https://eax.me/yesod-persistent/, свободный.
	 
	 
	 \bibitem{elm} An Introduction to Elm. [Электронный ресурс]. Режим доступа: https://guide.elm-lang.org/, свободный.
	 
\end{thebibliography}